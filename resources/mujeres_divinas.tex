
\section{Mujeres Divinas}

\noindent
\chordC
\chordAm
\chordDm
\chordGs

\vspace{1cm}

\begin{guitar}
	\begin{multicols}{2}

		Intro: [D7].......[G].......[D7].......[G]

		\hspace{1cm}[B7].......[Em].......[B7].......[Em].

		[Em]Hablando de mu[E7]jeres y trai[Am]ciones
		Se [B7]fueron consumiendo las bo[Em]tellas
		Pidieron que can[G]tara mis can[B7]ciones
		Y yo canté unas dos en contra de [Em]ellas


		De [Em]pronto que se a[E7]cerca un caba[Am]llero
		Su [B7]pelo ya pintaba algunas [Em]canas
		Me dijo "[G]le suplico compañ[B7]ero
		Que no hable en mi presencia de las [Em]damas"


		Le di[Am]je que no[D7]sotros simple[G]mente
		Ha[Am]blamos de lo [D7]mal que nos pa[G]garon

		Que si [B7]alguien opinaba dife[Em]rente
		Se[B7]ría porque jamás lo traicio[Em]naron
		Que si [B7]alguien opinaba dife[Em]rente
		Se[B7]ría porque jamás lo traicio[Em]naron


		\textit{Requinto (igual a intro)}


		Me [Em]dijo yo soy [E7]uno de los [Am]seres
		Que [B7]más ha soportado los fra[Em]casos
		Y siempre me de[G]jaron las mu[B7]jeres
		Llorando y con el alma hecha pe[Em]dazos


		Mas [Em]nunca les re[E7]procho mis heri[Am]das
		Se [B7]tiene que sufrir cuando se [Em]ama
		Las horas más her[G]mosas de mi [C7]vida
		Las he pasado al lado de una [Em]dama



		Pu[Am]diéramos [D7]morir en las can[G]tinas
		Y [Am]nunca logr[D7]aríamos olvi[G]darlas
		Mu[B7]jeres, oh mujeres tan di[Em]vinas
		No [B7]queda otro camino que ado[Em]rarlas
		Mu[B7]jeres, oh mujeres tan di[Em]vinas
		No [B7]queda otro camino que ado[Em]rarlas
	\end{multicols}
\end{guitar}
